\newprob{lq1}{
    一個物件 O 放在一片透鏡 $L$ 前,成像 I 如圖中所示。\\
    An object O is placed in front of lens $L$, the image I is shown in the figure below.
    \begin{figure}[h!]
        \centering
        \includegraphics[width=.9\linewidth]{d98n201n20d9d.png}
    \end{figure}
    \begin{parts}
        \part $L$ 是何種透鏡?解釋你的答案。\\
        what kind of lens is $L$? Explain your answer.\zh{2}
        \dlines{2}
        \clearpage
        \part 透過加入適當的光線,在圖中畫出透鏡的位置($L$)、主焦點的位置 (F) 並求透鏡焦距。\\
        By adding suitable ray(s) in the figure, indicate the locations of lens ($L$), principal focus (F) and write down the focal length of the lens.\zh{5}
        \vspace{.55cm}\par 焦距focal length: \fillin[][1.5in]

        \part 完成光線 r 折射後的光路。\\Complete ray r.\zh{1}
        \part 透鏡 $L$ 和透鏡 M 的形狀、大小相同,但透鏡 M 的折射率略高。 把透鏡 $L$ 換成透鏡 M 後,成像的放大率會如何改變?扼要解釋你 的答案。\\Lens $L$ and lens M have the same shape and size, but lens M has a slightly higher refractive index. How will the magnification of the image change when lens $L$ is replaced with lens M? Briefly explain your answer.\zh{2}
        \dlines{1}
    \end{parts}
}{}
\newprob{lq2}{
    把一件高15 cm的物體,放在凸透鏡前45 cm的 位置。下圖顯示兩條從物體發出的光線,及其中 一條折射線。\\An object of height 15 cm is placed 45 cm in front of a convex lens. The figure shows two light rays emitted from that object. One of the paths is as depicted.
    \bigskip\here{\includegraphics[width=0.6\linewidth]{dwqi0mid9d.png}}\bigskip
    \begin{parts}
        \part 試畫上餘下的折射線。\\Complete the path of the remaining light ray. \zh{2}
        \part 由此或以其他方法,求透鏡的焦距。\\Hence, or otherwise, find the focal length of the lens. \zh{1}
        \vspace{.55cm}\par 焦距focal length: \fillin[][1.5in]
        \bigskip
        \part 試指出像的本質,解釋你的答案。\\State the nature of the image, explain your answer.\zh{2}
        \dlines{1}

    \end{parts}
}{}
\newprob{lq4}{
    An object PQ is placed in front of lens X. A ray from Q is refracted by lens X as shown in the ray diagram below.
}{}
\newprob{lq3}{
    如圖所示,一位學生用透鏡觀看發光的字母「F」。\\An illuminated letter `F' is viewed through a lens as follow.

    \bigskip{\par\centering\includegraphics[width=0.7\linewidth]{cdcce.png}\par}\bigskip
    \begin{parts}
        \part 不論透鏡和字母「F」之間的距離怎樣改變,透鏡都不能產生實像。這是哪一種透鏡?\\The lens cannot form a real image of the letter no matter how the distance between them is varied. State the type of lens used.\zh{1}
        \ddlines{.5}
        \part 草繪學生看到的像。\\Sketch the image as seen by the observer.\zh{1}
        \bigskip\here{
            \includegraphics[width=0.2\linewidth]{9j290c3c0e92.png}
        }
        \clearpage\part 透鏡的焦距是 15 cm,形成的像與透鏡相距 10 cm。在下面的方格紙中繪畫光線圖,顯示像怎樣形成。\\The focal length of the lens is 15 cm and the image formed is 10 cm away from the lens. In the following graph paper, draw a ray diagram to show how the image is formed.\zh{3}
        {\par
            \centering
            \includegraphics[width=0.75\linewidth]{d8un293u1d3.png}
            \par}
        \part 如果學生把透鏡移近字母「F」,像距、像的方向會怎樣改變?\\If the lens is moved closer to the letter ‘F’, how will the image distance, the orientation and of the image change?\zh{2}
        \dlines{2}
    \end{parts}
}
{
    \begin{figure}
        \centering
        \includegraphics[width=1\linewidth]{dnu28923u8d3.png}
    \end{figure}
    \begin{figure}
        \centering
        \includegraphics[width=1\linewidth]{92nu0cu2982.png}
    \end{figure}
}



\newprob{mc1}{
    一個物件放在一片凹透鏡前 12 cm 處,產生的成像像距為 8 cm 。若把物
    件移至凹透鏡前 24 cm 處,產生的成像像距為\\
    A object is placed 12 cm in front of a concave lens, and the image distance is 8 cm. If the object is moved to a position 24 cm in front of the concave lens, the resulting image distance is
    \begin{tasks}
        \task 6 cm
        \task 12 cm
        \task 16 cm
        \task 18 cm
    \end{tasks}
}{}
\newprob{mc2}{
    利用光線箱照亮印有字母``F''的半透明屏 幕。把凸透鏡 $L$ 和屏幕 $S$ 放在適當的距離,以 便在 $S$ 上形成清晰、倒立及等大的成像。下列哪些陳述是正確的?\\A translucent screen printed with a letter ``F'' is illuminated by a ray box. A convex lens $L$ and a screen $S$ are placed so that an inverted sharp image of the same size as the object is produced. Which of the following statements is/are true?
    \bigskip\here{
        \centering\includegraphics[width=0.35\linewidth]{dedm22n3dnu23.png}
    }\bigskip

    \begin{statements}
        \task 若把 $L$ 略為向右移,成像將向左移。\\
        If $L$ is moved to the right through a short distance, the image will move to the left.
        \task 若把 $L$ 略為向上提,成像將向上移。\\
        If $L$ is raised through a short distance, the image will move upward.
        \task 若把 $L$ 略為向光線箱移動, $S$ 須移離光線 箱,以便再次捕捉成像。\\If $L$ is raised through a short distance, the image will move upward.
    \end{statements}
    \begin{tasks}
        \task 只有(1) \tab\tab (1) only
        \task 只有(1)和(3) \tab\tab (1) and (3) only
        \task 只有(2)和(3) \tab\tab (2) and (3) only
        \task (1), (2) 和 (3)\tab\tab (1), (2) and (3)
    \end{tasks}
}{C}
\newprob{mc3}{
    如圖顯示,物體藉凸透鏡形成一個倒置的像。 圖中亦繪畫了四條來自這物體的光線。哪條光線是不正確的?\\
    An object is placed in front of a convex lens and an inverted image is formed as shown. Four rays from the object are drawn. Which of them is/are incorrectly drawn?
    \bigskip\here{
        \centering
        \includegraphics[width=0.35\linewidth]{dedewdwed12dc4f43ge.png}
    }\bigskip

    \begin{tasks}
        \task 只有(1) \tab\tab (1) only
        \task 只有(4) \tab\tab (4) only
        \task 只有(2)和(4) \tab\tab (2) and (4) only
        \task (2), (3) 和 (4)\tab\tab (2), (3) and (4)
    \end{tasks}
}{C}
\newprob{mc4}{
    如圖顯示,兩條會聚的光線投射在一個主焦點 為 $F$ 的透鏡上。哪字母代表折射線所會聚的 一點?\\
    A pair of converging light rays strike a lens with focus $F$ as shown. Which of the letters represents the point where the rays will be converged?
    \bigskip\here{\includegraphics[width=0.4\linewidth]{imdqd9m01u32d-qage.png}}\bigskip
    \begin{tasks}
        \task $P$
        \task $Q$
        \task $R$
        \task $S$
    \end{tasks}
}{D}
\newprob{mc5}{
    如圖顯示一條光線,穿過凹透鏡 $L$ 後發生的 折射。哪字母可能是 $L$ 的主焦點?\\
    A ray of light is refracted by a concave lens $L$ as shown. Which of the letters can be the focus of $L$?
    \bigskip\here{\includegraphics[width=0.45\linewidth]{d8uxdn01ud892.png}}\bigskip
    \begin{tasks}
        \task $P$
        \task $Q$
        \task $R$
        \task $S$
    \end{tasks}
}{}
\newprob{mc6}{
    \topalignc{\includegraphics[width=0.4\linewidth]{d2d230dm9i23d9i23932age.png}}\bigskip\\
    如圖顯示,一個透鏡放在印有``PHYSICS'' 字樣的紙板上的情況。以下哪些改變可增加成像的尺寸?\\
    The diagram shows the result when a lens is held above a paper printed with the word ``PHYSICS''. Which of the following may increase the size of the image?

    \begin{statements}
        \task 增加透鏡的焦距\\increase the focal length
        \task 把透鏡的位置略為提高\\raise the lens higher
        \task 使用較大折射率的透鏡\\use a lens of larger refractive index
    \end{statements}
}{}
\newprob{mc7}{
    \bigskip\topalignc{\includegraphics[width=0.5\linewidth]{deudn098u2.png}}\bigskip
    \\把物體放在一個凸透鏡前,並前後移動。然 後記錄物距$v$ 和相應的像距$v$。上圖顯示 $1/u$ 和$1/v$的關係線圖。透鏡的焦距是多少?\\An object is moved in front of a convex lens. The object distance $u$ and the corresponding image distance $v$ are recorded. A graph of $1/u$ against $1/v$ is plotted as shown above. What is the focal length of the lens?
    \begin{tasks}
        \task 10 cm
        \task 15 cm
        \task 20 cm
        \task 25 cm
    \end{tasks}
}{}
\newprob{mc8}{
    % .\\\par \here{\includegraphics[width=0.35\linewidth]{dijdmioj230.png}}
    \topalignc{\includegraphics[width=0.35\linewidth]{dijdmioj230.png}}\bigskip
    \\物體沿一個凸透鏡的主軸前後移動。以上的 圖表顯示物距$u$和像距$v$的關係。在哪一點 上,像距最接近透鏡的焦距?\\An object is moved along the principal axis of a convex Lens. The graph above shows a plot of object distance $u$ against image distance $v$. At which of the above points is the image distance most close to the focal length of the lens?
}{}
\newprob{mc9}{
    一條光線通過兩塊透鏡,並入射線與出射線皆與 主軸平行,如圖。下列哪項正確?
    A light ray passes two lenses as shown. Both the incident and emergent rays are parallel to the principal axis. Which of the following statements is/are correct?
    \bigskip\here{\includegraphics[width=0.2\linewidth]{dqwdqwdun89qdu808n98.png}}\bigskip
    \begin{statements}
        \task 凸透鏡的焦距較凹透鏡長。\\The focal length of the convex lens is longer than that of the concave lens.
        \task 兩塊透鏡各自一個主焦點在凹透鏡右方重 疊。\\One of the foci of the convex lens and one of the foci of the concave lens overlap on the right of the concave lens.
        \task 即使入射線並非平行於主軸,它與出射線仍 然互相平行。\\The incident and the emergent rays are still parallel if the incident ray is not parallel to the principal axis.

    \end{statements}
    \begin{tasks}
        \task 只有(1) \tab\tab (1) only
        \task 只有(3) \tab\tab (3) only
        \task 只有(1)和(2) \tab\tab (1) and (2) only
        \task 只有(2)和(3) \tab\tab (2) and (3) only
    \end{tasks}
}{C}
\newprob{mc10}{
    把一塊透鏡放在書本前,如圖。\\A lens is placed above a book as shown.
    \bigskip\here{\includegraphics[width=0.25\linewidth]{dn092i3dn923d32e.png}}\bigskip
    下列哪項正確?\\Which ones are correct?
    \begin{statements}
        \task 透鏡是凹透鏡。\\It is a convex lens.
        \task 成像是虛像。\\ The image is virtual.
        \task 像距較透鏡的焦距短。\\Object distance is shorter than focal length.
    \end{statements}
    \begin{tasks}
        \task 只有(1)和(2) \tab\tab (1) and (2) only
        \task 只有(1)和(3) \tab\tab (1) and (3) only
        \task 只有(2)和(3) \tab\tab (2) and (3) only
        \task (1), (2) 和 (3)\tab\tab (1), (2) and (3)
    \end{tasks}
}{D}
\newprob{mc11}{
    一個物體通過透鏡成像,如圖。\\An object and its image formed by a lens are as shown.
    \bigskip\here{\includegraphics[width=0.5\linewidth]{di0m992imd092-d.png}}\bigskip
    透鏡的焦距是多少?\\What is the focal length of the lens?
    \begin{tasks}
        \task 1 cm
        \task 1.5 cm
        \task 4 cm
        \task 6 cm
    \end{tasks}
}{D}
\newprob{mc12}{
    把一根蠟燭放在牆壁前一段距離外。在兩者之間 放置一塊透鏡,並緩慢移動透鏡。當透鏡移至途 中兩點,均有清晰的像在牆壁上形成。蠟燭在兩 處的像高分別為50 cm 和8 cm。問蠟燭的高度 是多少?\\
    A candle is placed at a fixed distance in front of a wall. A lens is inserted and moved slowly between them. At two Particular positions, sharp images are formed on the wall. The heights of the images are 50 cm and 8 cm respectively. What is the height of the candle?
    \begin{tasks}
        \task 6.25 cm
        \task 20 cm
        \task 21 cm
        \task 29 cm
    \end{tasks}
}{B}
\newprob{mc13}{
    一束會聚光線射向凹透鏡,並在距離透鏡10 cm 外的 P 點會聚。已知透鏡焦距為4 cm。若移開 透鏡,光線會在 Q 點會聚。\\
    A convergent beam is incident on a concave lens of focal length 4 cm as shown. It converges at P, which is 10 cm from the lens. If the lens is taken away, the beam will converge at Q.
    \bigskip\here{\includegraphics[width=0.25\linewidth]{d9nd8u92309823.png}}\bigskip
    Q點距離透鏡多遠?\\How far is Q from the position of the lens?
    \begin{tasks}
        \task 2.5 cm
        \task 2.9 cm
        \task 4 cm
        \task 6.7 cm
    \end{tasks}
}{}
\newprob{mc14}{
    下圖顯示某個像的線性 放大率 $m$ 隨像距$v$ 的 變化。\\The graph shows how the linear magnification $m$ of an image varies with its distance from the lens $v$.
    \bigskip\here{\includegraphics[width=0.25\linewidth]{du8nu903d2.png}}\bigskip
    今使用另一塊焦距較長的透鏡。下列哪幅線圖正 確?(原有線圖以短虛線表示。)\\
    A lens of a longer focal length is used instead. Which of the following graphs is correct? (The original graph is shown by the dotted line.)
    \begin{tasks}
        (2)
        \task
        \topalign{\includegraphics[width=0.55\linewidth]{ddqwð3dun3298dge.png}}
        \task
        \topalign{\includegraphics[width=0.55\linewidth]{dn0892ud023.png}}
        \task
        \topalign{\includegraphics[width=0.55\linewidth]{dnu98ud298dun923d.png}}
        \task
        \topalign{\includegraphics[width=0.55\linewidth]{dud2u0n2ndu8dge.png}}
    \end{tasks}

}{}
\newprob{mc15}{
    現有一個固定的發光物件和一個固定的屏幕,物件和屏幕間的距離為 5 m。當一片焦距為 1.3 m 的透鏡放在物件前
    $x$ m 時,以下哪項是$x$ 的可能值?\\
    There is a fixed luminous object and a fixed screen, with a distance of 5 m between them. When a lens with a focal length of 1.3 m is placed at a distance of $x$ m in front of the object, which of the following is a possible value for $x$?

    \newprob{mc16}{
        現有一個固定的發光物件和一個固定的屏幕。當一片透鏡分別放在物件前
        1 m 和 4 m 時,屏幕上出現了高度分別為 $h_1$ 和 $h_2$ 的成像。求 $h_1$: $h_2$。\\
        There is a fixed luminous object and a fixed screen. When a lens is placed at distances of 1 m and 4 m in front of the object respectively, two images with heights of $h_1$ and $h_2$ appear on the screen. The task is to find the ratio $h_1:h_2$.
        \begin{tasks}
            \task $16:1$
            \task $4:1$
            \task $2:1$
            \task $1:2$
        \end{tasks}
    }{B}
    \newprob{mc17}{}{}
    \begin{tasks}
        \task $1.12$
        \task $1.88$
        \task $2.60$
        \task $4.41$
    \end{tasks}
}{D}